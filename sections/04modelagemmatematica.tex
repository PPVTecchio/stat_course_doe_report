\chapter{A Modelagem Matemática}
\label{chap:a_modelagem_matematica}

%! Escrever um pouco sobre o que foi feito

\section{Regressão linear}
\label{sec:a_modelagem_matematica_modelo_linear}

A Regressão Linear é utilizada para analisar as relações entre o valor de uma variável resultante Y e uma ou mais variáveis X e suas interações. 

O objetivo desta técnica é obter um modelo matemático que seja capaz de representar um fenômeno, demonstrando mediante uma relação linear as relações entre a variável de entrada e a variável de saída. Vale destacar que após a criação do modelo, é possível estimar os valores da variável de resposta Y, utilizando valores conhecidos da variável preditora X.

A Regressão Linear modela uma variável Y como uma função matemática de um ou mais variáveis x, e então esse modelo de regressão pode ser utilizado para obter Y quando apenas o x é conhecido. A função matemática genérica é dada pela função:

Y = β1 + β2X + ϵ

% Figura

Onde o β1 é a interceptação da reta com o eixo vertical e β2 é a inclinação ou o coeficiente angular em relação à variável, eles são chamados de coeficientes de regressão. O parâmetro ϵ é o erro, ou seja, a porção de Y que o modelo de regressão não é capaz de explicar.

\section{Utilizando o modelo linear}
\label{sec:a_modelagem_matematica_utilizando_o_modelo_linear}

O modelo utilizado é um modelo linear múltiplo, para verificar o modelo matemático utilizando o código a seguir:

%! Colocar código

Neste modelo, são utilizados todos os parâmetros, de modo a verificar quais variáveis tem maior interferência sobre o modelo final e então fazer um modelo linear levando em conta estas variáveis que mais representam este modelo. Os coeficientes obtidos utilizando o modelo linear são mostrados a seguir.

% Colocar figura ou tabela

O valor β1, correspondente à interceptação da reta é mostrado na primeira linha. Os valores corespondentes à inclinação são mostrados na coluna "Estimate", visto que estes são os valores correspondentes a cada variável. Logo, o modelo pode ser medido pela seguinte equação:

Y = - 0.14125 + 0.85625 * Altura - 0.11 * Clipe + 0.05917 * AdesivoTopo + 0.03 * AdesivoLateral

Na equação, a modelagem do helicóptero é dada em função da altura, da presença do clipe e dos adesivos, tanto no topo quanto na lateral. Um aspecto que também deve ser levado em consideração é o p-valor dos coeficientes do modelo linear. O p-valor indica a hipótese deve ser rejeitada ou aceita, neste caso, a hipótese é que o preditor não é significativo para o modelo. Nesta avaliação, para verificar se os preditores são significativos ou não, é verificar se os valores de p são menores que 0,05.

Na análise do modelo, existem dois p-valores de dois preditores que são muito abaixo de 0.05, que são Altura e Clipe. Estes pequenos valores indicam que provavelmente Altura e Clipe sejam uma ótima adição ao modelo. Os valores de p-valor para as variáveis AdesivoTopo e Adesivo lateral são 0.076 e 0.36, respectivamente. Esses valores indicam que existe 8\% e 36\% de chances que essas variáveis não sejam significativas para a regressão.

Residuos

Para testar a qualidade do ajuste do modelo, um parâmetro considerado é a observação dos resíduos, que são as diferenças entre os valores reais e os valores preditos. Como a regressão linear é o processo de traçar uma reta através dos dados em um diagrama de dispersão, a reta principal representa os valores preditos. Os resíduos são a mínima distância entre os valores preditos e os valores reais, representados pelas retas vermelhas na figura. 

% Colocar figura








%! Escrever sobre código usado para criar o modelo linear, como este é capaz de utilizar não linearidades

\section{O modelo resultante}
\label{sec:a_modelagem_matematica_o_modelo_resultante}

%! Escrever sobre o resultado encontrado e o que ele nos informa