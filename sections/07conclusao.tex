\chapter{CONCLUSÃO}
\label{chap:conclusao}

Este trabalho apresentou os resultados do experimento de modelagem matemática de um helicóptero de papel. Com base na teoria de \ac{DOE}, foram escolhidas possíveis alterações no modelo físico do helicóptero de forma a investigar o efeito destas alterações no tempo de vôo do helicóptero. As quatro variáveis investigadas foram a altura de lançamento, a presença de clipe de papel na sua base, de adesivo no topo conectando as duas hélices e de adesivo em ambos os lados do corpo.

Através da análise realizada, foi possível gerar um modelo linear relacionando o tempo de vôo com a altura de lançamento e com a presença do clipe de papel, sendo a primeira diretamente proporcional e a segunda inversamente proporcional. Logo, o modelo indica que quanto mais elevado for o ponto de partida, maior será o tempo de vôo, por outro lado a presença do clipe de papel diminui o tempo de vôo. Tais resultados estão de acordo com o esperado, uma vez que a elevação da altura aumenta a distância a ser percorrida e a presença do clipe aumenta a massa do helicóptero.

Além do modelo matemático encontrado estar condizente com o esperado, os resultados da análise de correlação entre os valores preditos pelo modelo e os encontrados no experimento definido no Capítulo \ref{chap:segundo_experimento}, tal como visto no Capítulo \ref{chap:verificacao_do_modelo}, foram excelentes com valor de correlação de 99\%. Estes resultados evidenciam a aplicabilidade da teoria de \ac{DOE} e de modelagem matemática através da aplicação das técnicas de aproximação linear com fatores possivelmente não lineares.