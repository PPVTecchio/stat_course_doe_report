\chapter{Segundo Experimento}
\label{chap:segundo_experimento}

%! Escrever um pouco sobre o que foi feito

Analogamente aos experimentos realizados no Capítulo \ref{chap:primeiro_experimento}, foi gerado um conjunto de experimentos com o intuito de verificar o modelo matemático proposto no Capítulo \ref{chap:a_modelagem_matematica}. Os novos experimentos buscam então avaliar o modelo linear proposto com base apenas nas variáveis altura do lançamento e presença do clipe no helicóptero de papel.

\section{Criando o experimento no R}
\label{sec:segundo_experimento_criando o experimento_no_R}

%! Escrever sobre código usado para criar o experimento, a sequencia de testes, no R. Qual o motivo de usar isso?

O código originalmente criado para a formulação do primeiro conjunto de experimentos, disposto no Apêndice \ref{sec:app_codigo_geracao_do_experimento_de_modelagem}, foi modificado de forma a gerar o novo conjunto de experimentos, Apêndice \ref{sec:app_codigo_geracao_do_experimento_para_teste_do_modelo}. A sequência de variações dos parâmetros do helicóptero de papel a serem realizadas neste experimento está disposta na Tabela \ref{tab:experimentos_para_teste_do_modelo}.

\begin{table}[H]
  \centering
  \caption{Experimentos para teste do modelo.}
  \resizebox{0.25\textwidth}{!}{%
  \begin{tabular}{c|c}
  \textbf{Altura} & \textbf{Clipe} \\ \hline
  1.7             & -              \\ \hline
  1.7             & +              \\ \hline
  1.0             & -              \\ \hline
  1.0             & +              \\ \hline
  2.2             & +              \\ \hline
  1.5             & +              \\ \hline
  1.5             & -              \\ \hline
  2.2             & -
  \end{tabular}%
  }
  \legend{Fonte: Autores.}
  \label{tab:experimentos_para_teste_do_modelo}
\end{table}

\section{A coleta de dados}
\label{sec:segundo_experimento_a_coleta_de_dados}

Os dados coletados estão dispostos na Tabela \ref{tab:dados_experimentais_teste_modelo}. A coleta destes dados seguiu o mesmo procedimento do Experimento 1, apresentado na Seção \ref{sec:primeiro_experimento_a_coleta_de_dados}.

%! Escrever sobre como foi feita a coleta de dados e indicar a tabela resultante