\chapter{Tabelas de dados dos experimentos}
\label{chap:app_tabelas_de_dados_dos_experimentos}

Os dados experimentais encontrados para o tempo de vôo do modelo de helicóptero de papel estão apresentados a seguir. A seção Experimento 1 apresenta os resultados obtidos a partir dos experimentos gerados na Seção \ref{sec:primeiro_experimento_criando o experimento_no_R} e apresentados na Tabela \ref{tab:experimento_modelagem_helicoptero_papel}. Analogamente, a seção Experimento 2 apresenta os dados encontrados para o experimento da Seção \ref{sec:segundo_experimento_criando o experimento_no_R} e apresentados na Tabela \ref{tab:experimentos_para_teste_do_modelo}.

\section{Experimento 1}
\label{sec:app_experimento_1}

\begin{table}[H]
  \caption{Dados experimentais para modelagem.}
  \resizebox{\textwidth}{!}{%
  \begin{tabular}{c|c|c|c|c|c}
  \textbf{Teste1\_Mateus} & \textbf{Teste1\_Pedro} & \textbf{Teste2\_Mateus} & \textbf{Teste2\_Pedro} & \textbf{Teste3\_Mateus} & \textbf{Teste3\_Pedro} \\ \hline
  0.90 & 1.05 & 0.73 & 1.03 & 1.12 & 0.98 \\ \hline
  1.61 & 1.61 & 1.71 & 1.73 & 1.59 & 1.79 \\ \hline
  1.22 & 1.14 & 1.18 & 1.05 & 1.02 & 0.93 \\ \hline
  1.61 & 1.47 & 1.52 & 1.59 & 1.54 & 1.55 \\ \hline
  0.86 & 0.93 & 0.84 & 0.76 & 0.88 & 0.92 \\ \hline
  1.60 & 1.67 & 1.78 & 1.66 & 1.82 & 1.85 \\ \hline
  1.70 & 1.71 & 1.35 & 1.25 & 1.56 & 1.66 \\ \hline
  0.95 & 0.98 & 1.13 & 0.92 & 0.94 & 1.05 \\ \hline
  1.98 & 2.00 & 1.75 & 1.63 & 1.52 & 1.56 \\ \hline
  1.02 & 1.02 & 1.04 & 1.02 & 0.96 & 0.91 \\ \hline
  1.47 & 1.66 & 1.63 & 1.62 & 1.82 & 1.65 \\ \hline
  1.55 & 1.45 & 1.60 & 1.72 & 1.71 & 1.60 \\ \hline
  1.69 & 1.74 & 1.79 & 1.83 & 1.53 & 1.60 \\ \hline
  1.15 & 1.14 & 0.94 & 1.10 & 1.20 & 1.15 \\ \hline
  0.82 & 0.92 & 0.83 & 0.80 & 0.76 & 0.76 \\ \hline
  0.93 & 0.73 & 0.79 & 0.81 & 0.92 & 0.92
  \end{tabular}%
  }
  \legend{Fonte: Autores.}
  \label{tab:dados_experimentais_para_modelagem}
\end{table}

\section{Experimento 2}
\label{sec:app_experimento_2}

\begin{table}[H]
  \caption{Dados experimentais para teste do modelo.}
  \resizebox{\textwidth}{!}{%
  \begin{tabular}{c|c|c|c|c|c}
  \textbf{Teste1\_Mateus} & \textbf{Teste1\_Pedro} & \textbf{Teste2\_Mateus} & \textbf{Teste2\_Pedro} & \textbf{Teste3\_Mateus} & \textbf{Teste3\_Pedro} \\ \hline
  1.40 & 1.40 & 1.52 & 1.43 & 1.27 & 1.39 \\ \hline
  1.28 & 1.22 & 1.12 & 1.12 & 1.30 & 1.30 \\ \hline
  0.69 & 0.60 & 0.74 & 0.82 & 0.72 & 0.68 \\ \hline
  0.55 & 0.63 & 0.43 & 0.48 & 0.51 & 0.60 \\ \hline
  1.57 & 1.50 & 1.58 & 1.65 & 1.53 & 1.69 \\ \hline
  1.25 & 1.10 & 0.95 & 0.90 & 1.13 & 1.03 \\ \hline
  1.18 & 1.25 & 1.36 & 1.36 & 1.22 & 1.32 \\ \hline
  1.67 & 1.65 & 1.95 & 1.85 & 1.83 & 1.85
  \end{tabular}%
  }
  \legend{Fonte: Autores.}
  \label{tab:dados_experimentais_teste_modelo}
\end{table}